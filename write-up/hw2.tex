\documentclass[psamsfonts,onesided,10pt]{amsart}
\usepackage{goodstyle}


%opening
\title{HW 2 - Write Up}
\author{Robin Belton, Daniel Laden, Jiahui Ma,  Badr Zerktouni}

\begin{document}

\maketitle

\section{System Usage}
\subsection{System Setup}
This system was written with Python version 3.7. The system depends on the following python packages.

\begin{itemize}
    \item pandas
    \item numpy
    \item itertools 
    \item random
\end{itemize}

\subsection{Running the System}
This system uses Iris Data csv located at \path{https://github.com/jiahuiblair/CSCI-550-HW2/iris.csv} 
to test clustering and assessment algorithms. All python functions are included in the \path{main.py} file \todo{create this}

All functions can be run in the command line as follows:

\begin{description}
\item[Synthetic Data] To generate synthetic data where $(x,y)$ points are sampled from 
non-overlapping rectangular regions, run \textsc{SyntheticRectData}$(k,n,r)$ where $k$ is the 
number of points to sample from $n$ non-overlapping rectangular regions in $\mathbb{R}^2$. 
The noise parameter, $r$  is the number of noise points to add to the data that is not in the interior 
of any rectangular region. The output will be a data frame with attributes, `X', `Y', and `Label'. 
`X' and `Y' denote the $x-,y-$coordinates of the data point and the label denotes which rectangle 
the point was sampled from. A label of zero means the point is added noise.
\item[$k$-means] To cluster a two-dimensional dataframe $(x,y)$ through the $k$-means 
clustering algorithm, run \textsc{kmean}$(D,k,e)$ where $D$ is a three-dimensional dataframe with 
$x$, $y$ and a label column with NAN values, $k$ is the number of clusters specified by the user, 
and $e$ is the mean distance limitation between the most updated centroids and their previous 
centroids. The output of the \textsc{kmean}$(D,k,e)$ is a three-dimensional dataframe with $x$, $y$, 
and corresponding updated cluster label.
\item[DBSCAN] \todo{}
\item[Purity]  To compute the purity of a clustered data set, $C$, to a given partition, $D$ run 
\textsc{Purity}$(D, C, n)$. $C$ and $D$ need to be data frames with 
attributes `X',  `Y', and `Label'. Additionally, this function is written for data that has been 
classified in three clusters where the labels of the clusters are either 1, 2, or 3. 
\item[Silhouette] To conduct internal assessment, run \textsc{internalassessment}$(D,k)$ 
where $D$ is a three-dimensional dataframe with $x$, $y$, and corresponding updated cluster 
labels, and $k$ is the number of clusters in the clustering. The output of this function is the 
average silhouette coefficient value of each points $(x,y)$ silhouette coefficients. 
\end{description}
 
\section{Example of Input/Output}
\begin{description}
\item[Synthetic Data] We run the SyntheticRectData for sampling 30 points from 3 
rectangular regions with 10 noise points. We also plot the results.
\begin{verbatim}
synth = SyntheticRectData(30,3,10)
plt.plot(synth['X'][0:29], synth['Y'][0:29], 'ro', #plot results
         synth['X'][30:59], synth['Y'][30:59], 'bo', 
         synth['X'][60:89], synth['Y'][60:89], 'go',
         synth['X'][90:99], synth['Y'][90:99], 'yo')
plt.show()   
\end{verbatim}
\begin{figure}[H]
    \centering
    {\includegraphics[width=.4\textwidth]{images/synth.png}} \\
    \caption{Plotting output of synth. Red, blue, and green represent the three rectangular regions. 
Yellow points are the added noise.}
\end{figure}
\item[$k$-means] Below we run the $k$-means algorithm with a dataframe (Sepal Length, Sepal Width, Label), $k$ equals to 3, and $e$ is 0.05. 
\begin{verbatim}
k=3
e=0.05
km=the dataframe
kmean(km,k,e)
\end{verbatim}
\begin{figure}[H]
    \centering
    {\includegraphics[width=.3\textwidth]{images/kmeans.png}} \\
    \caption{$k$-means output when applied to Sepal Length and Sepal Width of the Iris dataset.}
\end{figure}

\item[DBSCAN] \todo{}
\item[Purity] Below we first produce the ground truth data for the Iris data set using Sepal Length 
and Sepal Width as the X and Y attributes. We then apply purity to the $k$-means output when $e=0.05$. 
\begin{verbatim}
D = iris[['Sepal Length', 'Sepal Width', 'Species']]
ground_truth = np.zeros((len(D.iloc[:,0]),3))
ground_truth[:,0] = D.iloc[:,0]
ground_truth[:,1] = D.iloc[:,1]
for i in range(len(D.iloc[:,0])):
    if D.iloc[i,2]=='Iris-setosa':
        ground_truth[i,2] = 1
    elif D.iloc[i,2]=='Iris-versicolor':
        ground_truth[i,2] = 2
    else:
        ground_truth[i,2] = 3
groundtruth = pd.DataFrame({'X':ground_truth[:,0], 'Y':ground_truth[:,1], 
                 'Label':ground_truth[:,2]})       
# Compute purity with Kmeans output, km (need to run k-means clustering function to get km)
Purity(groundtruth, km)
# output = 0.66
\end{verbatim}
\item[Silhouette] We run the internal assessment with the output dataframe from the 
$k$-means clustering algorithm where $k=3$. The ouput data frame from $k$-means is denoted by $km$.
\begin{verbatim}
internalassessment(km,k)
The Silhouette Coeff is 
0.27760826327927407
\end{verbatim}
\end{description}

\section{Exploring Datasets}
In this section we describe our findings from Part 4. We compared $k$-means to DBSCAN on the 
Iris data set using the assessment measures of Purity and Silhouette. Furthermore, we explored 
how varying the number of sample points and noise points affect the purity of $k$-means. 
\todo{insert how $k$-means did compared to DBSCAN}
Using our program to generate synthetic data that is drawn from three rectangles, we explored 
how increasing the sample size affected the purity of the $k$-means clustering. We found that 
keeping the noise constant while increasing the sample size did not appear to have an affect on 
purity. See Table 1. We repeated this small experiment several times and got similar results. 

\vspace{1ex}
\begin{center}
\begin{tabular}{ |c|c|c| } 
 \hline
\textbf{Sample} & \textbf{Noise} & \textbf{Purity}\\ 
40 & 5 & 0.648 \\ 
80 & 5 & 0.649 \\ 
120 & 5 & 0.611 \\ 
160 & 5 & 0.643 \\ 
200 & 5 & 0.648 \\ 
 \hline
\end{tabular}\\
\textbf{Table 1.} Purity assessments applied to $k$-means clustering of the synthetic data for 
different sample sizes.
\end{center}
\vspace{1ex}

Next, we used our program to generate synthetic data that is drawn from three rectangles to 
explore how increasing the noise affects the purity of the $k$-means clustering. We found that 
as noise increased, purity tended to decrease. See Table 2. Intuitively this makes sense since the 
noise points will be assigned to a cluster in $k$-means. However, by design, the noise points 
don't belong to any of the rectangles. Hence, the noise points will always negatively impact the purity measure. 

\vspace{1ex}
\begin{center}
\begin{tabular}{ |c|c|c| } 
 \hline
\textbf{Sample} & \textbf{Noise} & \textbf{Purity}\\ 
40 & 5 & 0.656 \\ 
40 & 10 & 0.59 \\ 
40 & 15 & 0.548 \\ 
40 & 20 & 0.59 \\ 
40 & 50 & 0.49 \\ 
 \hline
\end{tabular}\\
\textbf{Table 2.} Purity assessments applied to $k$-means clustering of the synthetic data for 
different amounts of noise.
\end{center}
\vspace{1ex}

\end{document}
